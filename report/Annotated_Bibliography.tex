\documentclass[12pt]{article}
\usepackage{url}
\usepackage{tikz}
\usepackage[utf8]{inputenc} %to get the acute and grave
\usepackage[T1]{fontenc}
\usepackage{hyperref}
\begin{document}
\begin{center}
   {\Large\textbf{Annotated Bibliography on Games for Teaching Children}}
\medskip

   { last edited: \today
   }
\end{center}

\section{Related Work} 

\begin{itemize}
\item\cite{paraskeva2010multiplayer} Covers the possible use of multiplayer concepts in educational games. Has a lot on aspects of games that may lead to better engagement regardless of whether or not the game is educational in the hopes of applying similar concepts to educational games. States that educational games should be designed in a way that prevents indefinite play.

\item\cite{holt2002expanding} Research on games for teaching children has received significant attention where concerning sports (primarily in contrast with drill training). This isn't directly relevent to what I'm researching as it's more concerned with teaching children \emph{to} play games rather than \emph{with} games, but it may provide some insight.

\item\cite{berns2021virtualreality}

\item\cite{dasilveira2021eguess}

\item\cite{kraiger1993application} On training evaluation, which initially sounds tangential, but the focus on the learning aspect of training is what makes this relevent. It breaks it up into learning and transfer issues, but in this case all I'm really interested in is learning issues. The researchers to classify learning outcomes as more than "changes in verbal knowledge or behavioral capacities," and intend to move towards a "conceptually based classification scheme." Their categories are "cognitive, skill-based, and affective." Cognitive is essentially knowledge and knowledge organization, skill-based is procedures (automatic or not), and affective is described as "attitudinal or motivational." They propose tests for each category of learning.
\end{itemize}


\section{Other Work}

\begin{itemize}
\item\cite{squire2008designing} Starts by dunking on poor kids and then immediately claims that any twelve year old student has regular access to a computer. Additionally it believes that Civilization III is an accurate representation of world history. Research is a little too old for what it says.

\item\cite{lopez2021robotics}
\end{itemize}

\nocite{*}

\bibliographystyle{alpha}
\bibliography{ReportBibliography}
\end{document}